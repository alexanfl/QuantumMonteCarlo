\documentclass[../main.tex]{subfiles}
 
\begin{document}
\begin{appendices}
%\section{Appendix}

URL to github folder with source files etc.\\
\url{https://github.com/christianfleischer/Quantum-Dot-Project}

\section{Calculations of Closed Form Expressions}
\subsection{Two-body quantum dots}\label{sec:ClosedFormTwo}
To find an expression for the local energy we need to find the Laplacian of the trial wave function
\begin{equation}\label{eq:trial}
   \psi_{T}({\bf r_1},{\bf r_2}) = 
   C\exp{\left(-\alpha\omega(r_1^2+r_2^2)/2\right)}
   \exp{\left(\frac{ar_{12}}{(1+\beta r_{12})}\right)}.
\end{equation}
We define 
\begin{align}
    u =& -\alpha\omega(r_1^2+r_2^2)/2\\
    v =& \frac{ar_{12}}{(1+\beta r_{12})}
\end{align}
Then we have 
\begin{align}
    \nabla_1 \exp(u) = -\alpha\omega \mathbf{r}_1\exp(u).
\end{align}
For $\nabla_1 \exp(v)$ we look at the first component of the gradient
\begin{align}\label{eq: chainDeriv}
\begin{split}
    \frac{\partial}{\partial x_1} \exp(v(r_{12})) =& \frac{\partial r_{12}}{\partial x_1}\frac{\partial}{\partial r_{12}} \exp(v(r_{12}))\\
    =& \frac{(x_1 - x_2)}{r_{12}} \frac{\partial}{\partial r_{12}} \exp(v(r_{12}))\\
    =& \frac{(x_1 - x_2)}{r_{12}} \frac{a}{(1+\beta r_{12})^2}\exp(v(r_{12})),
\end{split}
\end{align}
which gives 
\begin{align}
    \nabla_1 \exp(v) = \frac{(\mathbf{r}_1 - \mathbf{r}_2)}{r_{12}} \frac{a}{(1+\beta r_{12})^2}\exp(v).
\end{align}
For the second particle we have 
\begin{align}
    \nabla_2 \exp(u) =& -\alpha\omega \mathbf{r}_2\exp(u)\\
    \nabla_2 \exp(v) =& -\nabla_1 \exp(v).
\end{align}
The gradients for the full wave function are then 
\begin{align}
    \nabla_1 \psi_T =& \left(-\alpha\omega\mathbf{r}_1 + \frac{(\mathbf{r}_1 - \mathbf{r}_2)}{r_{12}} \frac{a}{(1+\beta r_{12})^2}\right)C\exp(u)\exp(v)\\
    \nabla_2 \psi_T =& \left(-\alpha\omega\mathbf{r}_2 - \frac{(\mathbf{r}_1 - \mathbf{r}_2)}{r_{12}} \frac{a}{(1+\beta r_{12})^2}\right)C\exp(u)\exp(v).
\end{align}
To find the Laplacian we need 
\begin{align}
    \nabla_1 (-\alpha\omega\mathbf{r}_1) = -\alpha\omega d = -2\alpha\omega,
\end{align}
where $d$ is the number of dimensions, in our case $d=2$. We also need 
\begin{align}
\begin{split}
    \nabla_1 \left(\frac{(\mathbf{r}_1 - \mathbf{r}_2)}{r_{12}} \frac{a}{(1+\beta r_{12})^2}\right) =& \frac{a}{(1+\beta r_{12})^2}\nabla_1 \frac{(\mathbf{r}_1 - \mathbf{r}_2)}{r_{12}} + \frac{(\mathbf{r}_1 - \mathbf{r}_2)}{r_{12}} \nabla_1 \frac{a}{(1+\beta r_{12})^2}\\
    =& \frac{d-1}{r_{12}}\frac{a}{(1+\beta r_{12})^2} + \frac{(\mathbf{r}_1 - \mathbf{r}_2)}{r_{12}}\frac{(\mathbf{r}_1 - \mathbf{r}_2)}{r_{12}}\frac{\partial}{\partial r_{12}} \frac{a}{(1+\beta r_{12})^2}\\
    =& \frac{1}{r_{12}} \frac{a}{(1+\beta r_{12})^2} - \frac{2a\beta}{(1+\beta r_{12})^3}.
\end{split}
\end{align}
We also have 
\begin{align}
    \nabla_2 (-\alpha\omega\mathbf{r}_2) = \nabla_1 (-\alpha\omega\mathbf{r}_1),
\end{align}
and 
\begin{align}
    \nabla_2 \left(-\frac{(\mathbf{r}_1 - \mathbf{r}_2)}{r_{12}} \frac{a}{(1+\beta r_{12})^2}\right) = \nabla_1 \left(\frac{(\mathbf{r}_1 - \mathbf{r}_2)}{r_{12}} \frac{a}{(1+\beta r_{12})^2}\right)
\end{align}
We end up with the Laplacians 
\begin{align}
    \frac{\nabla_1^2\psi_T}{\psi_T} =& \left(-2\alpha\omega + \frac{1}{r_{12}} \frac{a}{(1+\beta r_{12})^2} - \frac{2a\beta}{(1+\beta r_{12})^3} + \left(-\alpha\omega\mathbf{r}_1 + \frac{(\mathbf{r}_1 - \mathbf{r}_2)}{r_{12}} \frac{a}{(1+\beta r_{12})^2}\right)^2\right)\\
    \frac{\nabla_2^2\psi_T}{\psi_T} =& \left(-2\alpha\omega + \frac{1}{r_{12}} \frac{a}{(1+\beta r_{12})^2} - \frac{2a\beta}{(1+\beta r_{12})^3} + \left(-\alpha\omega\mathbf{r}_2 - \frac{(\mathbf{r}_1 - \mathbf{r}_2)}{r_{12}} \frac{a}{(1+\beta r_{12})^2}\right)^2\right).
\end{align}
The sum of the Laplacians in the Hamiltonian is then
\begin{align}
\begin{split}
    \frac{\nabla_1^2\psi_T}{\psi_T} + \frac{\nabla_2^2\psi_T}{\psi_T} = &-4\alpha\omega + \frac{1}{r_{12}}\frac{a}{(1+\beta r_{12})^2} - \frac{2a\beta}{(1 + \beta r_12)^3}\\
    &+ \alpha^2\omega^2(r_1^2 + r_2^2) + \frac{2a^2}{(1+\beta r_{12})^4} - \alpha\omega r_{12}\frac{2a}{(1+\beta r_{12})^2}.
\end{split}
\end{align}
The complete expression for the local energy is then 
\begin{align}
\begin{split}
    E_L =& \frac{1}{\psi_T}H\psi_T\\
    =& 2\alpha\omega + \frac{1}{r_{12}}\frac{a}{(1+\beta r_{12})^2} - \frac{2a\beta}{(1 + \beta r_12)^3} - \frac{1}{2}\alpha^2\omega^2(r_1^2 + r_2^2) \\
    &- \frac{a^2}{(1+\beta r_{12})^4} + \alpha\omega r_{12}\frac{a}{(1+\beta r_{12})^2} + \frac{1}{2}\omega^2(r_1^2 + r_2^2) + \frac{1}{r_{12}}\\
    =& 2\alpha\omega + \frac{1}{2}\omega^2(r_1^2 + r_2^2)(1 - \alpha^2) - \frac{2a\beta}{(1 + \beta r_12)^3} \\
    &- \frac{a^2}{(1+\beta r_{12})^4} + \left(\alpha\omega r_{12} + \frac{1}{r_{12}}\right)\frac{a}{(1+\beta r_{12})^2} + \frac{1}{r_{12}}.
\end{split}
\end{align}

\subsection{Many-body quantum dots}\label{sec:ClosedFormMany}
These calculations follow Ref.~\cite{FYS4411-Slides}.
In the many-body case we have the trial wave function 
\begin{align}
    \psi_{T}({\bf r_1},{\bf r_2},\dots, {\bf r_N}) = 
   Det\left(\phi_{1}({\bf r_1}),\phi_{2}({\bf r_2}),
   \dots,\phi_{N}({\bf r_N})\right)
   \prod_{i<j}^{N}\exp{\left(\frac{a r_{ij}}{(1+\beta r_{ij})}\right)}, 
\end{align}
where $Det$ is a Slater determinant, and the single-particle wave functions
are the harmonic oscillator wave functions given by
\begin{align}
    \phi_{n_x,n_y}(x,y) = A H_{n_x}(\sqrt{\omega}x)H_{n_y}(\sqrt{\omega}y)\exp{(-\omega(x^2+y^2)/2}.
\end{align}
$A$ is a normalization constant, while the functions $H_{n_x}(\sqrt{\omega}x)$ are Hermite polynomials. For $N=6$ electrons we need the Hermite polynomials for $n_x = 0,1$ and $n_y = 0,1$, for $N=12$ we need to include the $n_x,n_y = 2$ Hermite polynomials, and for $N=20$ we also need the Hermite polynomials for $n_x,n_y = 3$. When evaluating the trial wave function, the calculation of the gradient and the Laplacian of an $N$-particle Slater determinant is likely to be most time-consuming. This is because we have to differentiate with respect to all spatial coordinates of all electrons. We can improve the efficiency of the calculation by moving only one electron at the time. When we then differentiate the Slater determinant with respect to a given coordinate of that electron, only one row in the corresponding Slater matrix is changed. This means that we don't have to recalculate the entire determinant at every Metropolis step. Instead we use an algorithm which requires us to keep track of the inverse of the Slater matrix. 

The matrix elements of the Slater matrix $\hat{D}$ are given by 
\begin{align}
    d_{ij} = \phi_j(x_i),
\end{align}
where $\phi_j({\bf r_i})$ is a single particle wave function. $x_i$ is one of the spatial coordinates of the given particle, while $j$ indicates the quantum numbers ($n_x$ and $n_y$ in our case).
The inverse of $\hat{D}$ can be expressed by its determinant $|\hat{D}|$, and its cofactors $C_{ij}$ as follows
\begin{align}\label{eq: invdet}
    d_{ij}^{-1} = \frac{C_{ji}}{|\hat{D}|}, 
\end{align}
where the interchanged indices of $C_{ji}$ means that the cofactor matrix should be transposed. Assuming $\hat{D}$ is invertible we have
\begin{align}\label{eq: unity}
    \sum_{k=1}^N d_{ik}d_{kj}^{-1} = \delta_{ij}.
\end{align}
We define the ratio $R$, between $|\hat{D}({\bf r}^{\textrm{new}})|$ and $|\hat{D}({\bf r}^{\textrm{old}})|$, which by definition can be expressed as
\begin{align}\label{eq: R1}
    R \equiv \frac{|\hat{D}({\bf r}^{\textrm{new}})|}{|\hat{D}({\bf r}^{\textrm{old}})|} 
    = \frac{\sum_{j=1}^N d_{ij}({\bf r}^{\textrm{new}}) C_{ij}({\bf r}^{\textrm{new}})}{\sum_{j=1}^N d_{ij}({\bf r}^{\textrm{old}}) C_{ij}({\bf r}^{\textrm{old}})}.
\end{align}
If we move only one electron at a time, ${\bf r}^{\textrm{new}}$ and ${\bf r}^{\textrm{old}}$ differ only by the position of that one, $i$-th, electron, which means $\hat{D}({\bf r}^{\textrm{new}})$ and $\hat{D}({\bf r}^{\textrm{old}})$ differ only by the entries of the $i$-th row. The $i$-th row of a cofactor matrix $\hat{C}$ is independent of the entries in the $i$-th row of the corresponding matrix $\hat{D}$. In our case this means that the $i$-th row of $\hat{C}({\bf r}^{\textrm{new}})$ and $\hat{C}({\bf r}^{\textrm{old}})$ must be equal, so we have 
\begin{align}\label{eq: equalC}
    C_{ij}({\bf r}^{\textrm{new}}) = C_{ij}({\bf r}^{\textrm{old}})\quad \forall \ j \in \{1, \dots, N\}.
\end{align}
We use eq. (\ref{eq: invdet}) and eq. (\ref{eq: equalC}) with eq. (\ref{eq: R1}) in order to obtain
\begin{align}
    R = \frac{\sum_{j=1}^N d_{ij}({\bf r}^{\textrm{new}}) C_{ij}({\bf r}^{\textrm{old}})}{\sum_{j=1}^N d_{ij}({\bf r}^{\textrm{old}}) C_{ij}({\bf r}^{\textrm{old}})} 
    = \frac{\sum_{j=1}^N d_{ij}({\bf r}^{\textrm{new}}) d_{ji}^{-1}({\bf r}^{\textrm{old}})}{\sum_{j=1}^N d_{ij}({\bf r}^{\textrm{old}}) d_{ji}^{-1}({\bf r}^{\textrm{old}})},
\end{align}
where, by eq. (\ref{eq: unity}), the denominator of the rightmost expression is unity, and we end up with
\begin{align}
    R = \sum_{j=1}^N d_{ij}({\bf r}^{\textrm{new}}) d_{ji}^{-1}({\bf r}^{\textrm{old}}) = \sum_{j=1}^N \phi_j({\bf r}_i^{\textrm{new}}) d_{ji}^{-1}({\bf r}^{\textrm{old}}).
\end{align}
This means that if we only move the $i$-th electron, the ratio $R$ is given by the dot product between the vector, ($\phi_1({\bf r}_i^{\textrm{new}}),\dots,\phi_N({\bf r}_i^{\textrm{new}})$), of single particle wave functions evaluated at the new position, and the $i$-th column of the inverse matrix $\hat{D}^{-1}$ evaluated at the original position.

We need to maintain the inverse matrix, so if the new position ${\bf r}^{\textrm{new}}$ is accepted we need to use an algorithm for updating the inverse matrix. We start by updating all but the $i$-th column of $\hat{D}^{-1}$. For each column $j\neq i$, we calculate 
\begin{align}
    S_j = (\hat{D}({\bf r}^{\textrm{new}})\times \hat{D}^{-1}({\bf r}^{\textrm{old}}))_{ij} = \sum_{l=1}^N d_{il}({\bf r}^{\textrm{new}}) d_{lj}^{-1}({\bf r}^{\textrm{old}}), 
\end{align}
then we calculate the new elements of the $j$-th column of $\hat{D}^{-1}$ as follows
\begin{align}
    d_{kj}^{-1}({\bf r}^{\textrm{new}}) = d_{kj}^{-1}({\bf r}^{\textrm{old}}) - \frac{S_j}{R} d_{ki}^{-1}({\bf r}^{\textrm{old}}) \quad 
    \begin{array}{ll}
    \forall\ \ k\in\{1,\dots,N\}\\j\neq i
    \end{array}.
\end{align}
The last step is to update the $i$-th column of $\hat{D}^{-1}$ using the following equation
\begin{align}
    d_{ki}^{-1}(\mathbf{r}^{\mathrm{new}}) =
\frac{1}{R}\,d_{ki}^{-1}(\mathbf{r}^{\mathrm{old}})\quad
\forall\ \ k\in\{1,\dots,N\}.
\end{align}
Only the $i$-th row of the Slater matrix changes when differentiating the Slater determinant with respect to the coordinates of a single particle ${\bf r}_i$ as well, which means we can calculate the gradient and the Laplacian as follows 
\begin{align}\label{eq: gradSlater}
    \frac{\vec\nabla_i\vert\hat{D}(\mathbf{r})\vert}{\vert\hat{D}(\mathbf{r})\vert} =
    \sum_{j=1}^N \vec\nabla_i d_{ij}(\mathbf{r})d_{ji}^{-1}(\mathbf{r}) =
    \sum_{j=1}^N \vec\nabla_i \phi_j(\mathbf{r}_i)d_{ji}^{-1}(\mathbf{r})
\end{align}
and
\begin{align}\label{eq: lapSlater}
    \frac{\nabla^2_i\vert\hat{D}(\mathbf{r})\vert}{\vert\hat{D}(\mathbf{r})\vert} =
    \sum_{j=1}^N \nabla^2_i d_{ij}(\mathbf{r})d_{ji}^{-1}(\mathbf{r}) =
    \sum_{j=1}^N \nabla^2_i \phi_j(\mathbf{r}_i)\,d_{ji}^{-1}(\mathbf{r}).
\end{align}
Therefore in order to calculate the derivatives of the Slater determinant, we only need the derivatives of the single particle wave functions and the elements of the inverse Slater matrix.

The expectation value of the kinetic energy for electron $i$ expressed in atomic units is 
\begin{align}
    \langle \hat{K}_i \rangle = -\frac{1}{2}\frac{\langle\Psi|\nabla_{i}^2|\Psi \rangle}{\langle\Psi|\Psi \rangle}, 
\end{align}
and we have that 
\begin{align}
    K_i = -\frac{1}{2}\frac{\nabla_{i}^{2} \Psi}{\Psi}.
\end{align}
To find the kinetic energy we need the Laplacian of the wave function. We define the Slater determinant part of the wave function as $\Psi_D$ and define the correlation part (Jastrow factor) as $\Psi_C$. The Laplacian is then 
\begin{align}\label{eq: TotalLaplacian}
\begin{split}
    \frac{\nabla^2 \Psi}{\Psi} & =  \frac{\nabla^2 ({\Psi_{D} \,  \Psi_C})}{\Psi_{D} \,  \Psi_C} = \frac{\nabla  \cdot [\nabla  {(\Psi_{D} \,  \Psi_C)}]}{\Psi_{D} \,  \Psi_C} = \frac{\nabla  \cdot [ \Psi_C \nabla  \Psi_{D} + \Psi_{D} \nabla   \Psi_C]}{\Psi_{D} \,  \Psi_C}\\
    &  =  \frac{\nabla   \Psi_C \cdot \nabla  \Psi_{D} +  \Psi_C \nabla^2 \Psi_{D} + \nabla  \Psi_{D} \cdot \nabla   \Psi_C + \Psi_{D} \nabla^2  \Psi_C}{\Psi_{D} \,  \Psi_C}\\
    & =  \frac{\nabla^2 \Psi_{D}}{\Psi_{D}} + \frac{\nabla^2  \Psi_C}{ \Psi_C} + 2 \frac{\nabla  \Psi_{D}}{\Psi_{D}}\cdot\frac{\nabla   \Psi_C}{ \Psi_C}.
\end{split}
\end{align}
From eq. (\ref{eq: TotalLaplacian}) we see that we need the gradient and Laplacian of both $\Psi_D$ and $\Psi_C$. For $\Psi_D$ the necessary expression are given by eq. (\ref{eq: gradSlater}) and eq. (\ref{eq: lapSlater}). We have that 
\begin{align}
    \Psi_{C}=\prod_{i< j}\exp{f(r_{ij})}= \exp{\left\{\sum_{i<j}\frac{ar_{ij}}{1+\beta r_{ij}}\right\}},
\end{align}
and by differentiating with the chain rule similarly to what we did in eq. (\ref{eq: chainDeriv}), we find that the gradient is 
\begin{align}
    \frac{ \nabla_k \Psi_C}{ \Psi_C }= \sum_{j\ne k}\frac{{\bf r}_{kj}}{r_{kj}} \frac{\partial f(r_{kj})}{\partial r_{kj}} = \sum_{j\ne k}\frac{{\bf r}_{kj}}{r_{kj}} f'(r_{kj}) = \sum_{j\ne k}\frac{{\bf r}_{kj}}{r_{kj}}\frac{a}{(1+\beta r_{kj})^2},
\end{align}
where 
\begin{align}
    f'(r_{kj}) = \frac{\partial}{\partial r_{kj}} f(r_{kj}).
\end{align}
To find the Laplacian we need to calculate
\begin{align}
    \frac{\nabla^2_k \Psi_C}{\Psi_C } = \frac{1}{ \Psi_C }\nabla_k\sum_{j\ne k}\frac{{\bf r}_{kj}}{r_{kj}} f'(r_{kj}) \Psi_C.
\end{align}
We follow the calculations from Ref.~\cite{FYS4411-Slides} and Ref.~\cite{FYS4411-LectureNotes}. Using the product rule we get
\begin{align}
    \frac{\nabla^2_k \Psi_C}{\Psi_C } = &\frac{1}{ \Psi_C }\sum_{j\ne k}\left(\frac{{\bf r}_{kj}}{r_{kj}} f'(r_{kj}) \nabla_k\Psi_C + \frac{{\bf r}_{kj}}{r_{kj}} \Psi_C \nabla_k f'(r_{kj}) + f'(r_{kj}) \Psi_C \nabla_k\frac{{\bf r}_{kj}}{r_{kj}}\right)\\
    = &\sum_{ij\ne k}\frac{({\bf r}_k-{\bf r}_i)({\bf r}_k-{\bf r}_j)}{r_{ki}r_{kj}}f'(r_{ki})f'(r_{kj})+
    \sum_{j\ne k}\left( f''(r_{kj})+\frac{d-1}{r_{kj}}f'(r_{kj})\right),
\end{align}
where $d$ is the number of dimensions, $d=2$ in our case. We end up with the following expression 
\begin{align}
    \frac{\nabla^2_k \Psi_C}{\Psi_C }=
    \sum_{ij\ne k}\frac{({\bf r}_k-{\bf r}_i)({\bf r}_k-{\bf r}_j)}{r_{ki}r_{kj}}\frac{a}{(1+\beta r_{ki})^2}
    \frac{a}{(1+\beta r_{kj})^2}+
    \sum_{j\ne k}\left(\frac{a}{r_{kj}(1+\beta r_{kj})^2}-\frac{2a\beta}{(1+\beta r_{kj})^3}\right).
\end{align}


\section{Program Structure}

The methods were implemented with a C++ program using object-orientation. 
The program consists of several classes responsible for different parts of the simulations. The classes are:
\begin{itemize}
    \item {\bf Hamiltonian:} A super-class for different Hamiltonians. The subclasses calculate the local energy for their specific Hamiltonian. Since calculating the kinetic energy with numerical differentiation is done the same for all Hamiltonians, this super-class is responsible for that. The subclasses are:
    \begin{itemize}
        \item {\bf HarmonicOscillator:} Calculates the local energy in the non-interacting case. [Project1]
        \item {\bf HarmonicOscillatorRepulsive:} Calculates the local energy in the interacting case. [Project1]
        \item \textbf{HarmonicOscillatorElectrons:} Calculates the local energy for quantum dots. [Project2]
    \end{itemize}
    \item {\bf WaveFunction:} A super-class for different wave functions. The subclasses evaluate their specific wave function and also calculate the gradient, the Laplacian and the derivative w.r.t. the variational parameter(s) $\alpha$ (and $\beta$) using the analytical expressions. The subclasses are:
    \begin{itemize}
        \item {\bf SimpleGaussian:} WaveFunction subclass for the non-interacting case. [Project1]
        \item {\bf RepulsiveGaussian:} WaveFunction subclass for the interacting case. [Project1]
        \item \textbf{TwoElectrons:} WaveFunction subclass for the two-body quantum dot case. [Project2]
        \item \textbf{ManyElectrons:} WaveFunction subclass for the many-body quantum dot case. This includes all Slater determinant functionality. [Project 2]
    \end{itemize}
    \item {\bf InitialState:} A super-class for different initial states. The subclasses set up the initial state. The subclasses are:
    \begin{itemize}
        \item {\bf RandomUniform:} Sets up an initial state with uniformly distributed particle positions.
    \end{itemize}
    \item {\bf Particle:} Responsible for creating particles and adjusting their positions.
    \item {\bf System:} Responsible for running the Monte Carlo simulation. It performs the Metropolis and Metropolis-Hastings algorithms.
    \item {\bf Sampler:} Responsible for sampling interesting quantities and computing averages. It is also responsible for providing the data to the user, both by printing to terminal and saving to file.
    \item {\bf SteepestDescent:} Responsible for optimizing variational parameters using the Steepest Descent method. Once the optimal parameter has been found, the System class is tasked with running a large Monte Carlo simulation.
    \item {\bf Random:} Responsible for generating pseudo-random numbers according to different distributions.
\end{itemize}

There is also a main program which sets the necessary parameters and makes calls to the classes to start the simulation. The code is also fully parallelized with MPI. An advantage to using an implementation like this is that it makes it easy to add functionality, like more wave functions, Hamiltonians and initial states, or alternative methods for optimizing variational parameters (e.g. the Conjugate Gradient method). The data analysis is done in Python, with the programs {\bf blocking.py}, {\bf density.py} and \textbf{performance.py}.
\end{appendices}
\end{document}