\documentclass[../main.tex]{subfiles}
 
\begin{document}

In this thesis we study quantum dots trapped in various confinement potentials. Quantum dots refer to artificial creation of quantum confinement of fermions. These systems of fermions have much in common with atoms, but they are designed and fabricated in the laboratory\cite{Manninen}. Since systems of quantum dots are artificially created, the shape of the systems can be tuned as needed. The high level of control over quantum dot systems make them ideal for studying quantum effects, such as tunneling and entanglement. 

In addition to their significance in theoretical quantum physics research, quantum dots also have a number of uses in various electronic devices. The ability to artificially create quantum dot systems makes it possible to finely tune their optical and electrical properties. As a result, quantum dots are interesting prospects in areas like: solar cell technology\cite{QDotSolar, QDotSolar2}, medical imaging techniques\cite{QDotMed, QDotMed2}, laser technology\cite{QDotLaser, QDotLaser2}, and quantum computing\cite{QDotComp, QDotComp2}.

Variational and Diffusion Monte Carlo methods can be used to study gound state energies of quantum dot systems\cite{QMC1, QMC2, QMC3, QDotBenchmarks}. The variational Monte Carlo method uses the variational principle to find an upper bound estimate to the ground state energy of the system. This is done by creating a trial wave function to find the expectation value of the Hamiltonian $\langle H \rangle$. This trial wave function is dependent on some variational parameters, and when doing variational Monte Carlo, we vary these parameters in order to minimize $\langle H \rangle$. When studying many-body systems, the trial wave function usually includes the determinant of a Slater matrix, and this Slater matrix is built up of the single particle wave functions of the system.

The single particle wave functions of the system depend on what type of external potential the fermions are confined in. The main focus of this thesis is to develop a Quantum Variational Monte Carlo solver that can be used to study quantum dot systems, without needing explicit expressions for the single particle wave functions of the systems. This is done by diagonalizing the single particle problem in the potential of interest, and then expanding the solutions in terms of harmonic oscillator basis functions. The harmonic oscillator potential is often used in variational Monte Carlo simulations, and the corresponding single particle wave functions are well known. Provided that the harmonic oscillator functions are a decent fit, we can use them to approximate the single particle wave functions of more complicated potentials, and use these approximations in place of explicit expressions when doing simulations.

For this thesis we have developed a variational Monte Carlo solver that is general for any given potential. The solver is programmed in C++ using object-orientation and polymorphism, in order to make it simple to add new types of potential. We have used a single harmonic oscillator potential to verify the implementation. Since we use harmonic oscillator basis functions, the single particle wave function approximations should provide the same results as using the explicit expressions. Furthermore, we have studied systems with a double harmonic oscillator potential and with a single finite square well potential. These potentials are more complicated and not perfect fits for the harmonic oscillator basis, but if we use enough basis functions, a good approximation should be achievable. An important constraint of the method we have used is the reduced computational efficiency as the number of basis functions increase. Thus it is of interest to see how many basis functions are needed to get good approximations for the more complicated potentials, especially as the number of particles in the system increases.

In this thesis we study electrons, but the variational Monte Carlo solver we have developed can also be used to study other types of fermions. An example would be so-called ultra-cold neutrons, which are neutrons confined in magnetic traps\cite{ultra-cold neutrons}.

Using this variational Monte Carlo solver, we have studied quantum dot systems with those three potentials and estimated ground state energies for various numbers of particles and harmonic oscillator frequencies. We have also looked at the one-body densities of the systems, and studied how the one-body density changes when going from a single harmonic oscillator potential well to a double harmonic oscillator potential well.

\section*{Thesis Structure}
The thesis is made up of three parts, where the first presents the theoretical aspects, the second part explains the implementation and code structure, and the final part provides results and discusses them. The full structure is as follows:

\begin{itemize}
    \item The first part consists of chapters 2-5, and provides an explanation of the theory behind variational Monte Carlo and the single particle wave function approximations. Chapter \ref{sec: VMC} deals with the variational principle and the algorithms needed to do variational Monte Carlo simulations. It also covers the method used to estimate the statistical error of the simulations. Chapter \ref{sec: Basis Functions} explains how the single particle wave function approximations are made. Chapter \ref{sec: HF} introduces the Hartree-Fock method and discusses how it can be used to improve on the work done in this thesis. Finally, Chapter \ref{sec: Systems} introduces the potentials we have used.
    
    \item Chapters 6-8 make up the second part, which explains the implementation. Chapter \ref{sec: ProgramStructure} explains the code structure and how the theoretical methods are implemented. In Chapter \ref{sec: Testing} tests of the code are provided, which are used to verify the implementation. Finally, Chapter \ref{sec: OptimizingPerformance} provides insight into the optimizations made in order to make the VMC simulation efficient with regards to computational cost.
    
    \item The final part, which consists of chapters 9-10, presents the results. The results are listed and discussed in Chapter \ref{sec: Results}. First we look at how the computational efficiency improved from the code optimization. Next we look at single harmonic oscillator systems, and compare our ground state energy results to benchmarks from both quantum Monte Carlo methods and other many-body methods. The one-body densities of the systems are also presented and discussed. Next, we provide similar results and discussion for the double harmonic oscillator well, and finally for the single finite square well. Chapter \ref{sec: Conclusions} concludes the thesis, and includes suggestions for further work. 
\end{itemize}

\end{document}